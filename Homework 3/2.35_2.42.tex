\documentclass{article}
\usepackage{fancyhdr}
\usepackage{amsmath,amssymb}
\usepackage{geometry}
\usepackage{datetime}
\usepackage{enumerate}
\usepackage{graphicx}

%Insert page formatting here
%\hoffset = -.5in
\voffset = -0.375in
%\textwidth = 6in
\textheight = 8in
\headheight = 24pt

\pagestyle{fancy}

\rhead{Peter Olson\\Student ID: $441666$}
\lhead{Math 3200\\Homework 1}
\chead{\today}
\cfoot{}

%\addtolength{\headwidth}{\marginparsep}
%\addtolength{\headwidth}{\marginparwidth}

%\renewcommand{\labelitemi}{$\diamond$}
\renewcommand{\implies}{\rightarrow}
\newcommand{\widespace}{\qquad \qquad \;}
\newcommand{\tret}{\\ \hline}
\newcommand{\fh}{\tfrac{1}{2}}
\newcommand{\deriv}[2]{\frac{d #1}{d #2}}
\newcommand{\pderiv}[2]{\frac{\delta #1}{\delta #2}}
\newcommand{\vr}{\vec{r}}
\newcommand{\at}{\text{ at }}
\newcommand{\var}{\text{Var}}

\begin{document}
\section*{Exercise 2.35}
Let $X$ be the outcome of a random draw from integers 1 to $N$.
\begin{enumerate}
	\item What is the probability mass function?
	\begin{align*}
		f(x) &= P(X=x)\\
		f(x) &= \frac{1}{N}
	\end{align*}
	\item Show that $E(X) = (N+1)/2$ and $\var(X) = (N^2-1)/12$.
	\begin{align*}
		E(X) &= \sum_x (x)f(x)\\
		&= \frac{1}{N} \sum_x x\\
		&= \left( \frac{1}{N} \right)\left( \frac{N(N+1)}{2} \right)\\
		E(X) &= \frac{N+1}{2}\\
		\var(X) &= E(X-\mu)^2\\
		&= \sum_x (x-\mu)^2 f(x)\\
		&= \frac{1}{N} \sum_x \left(x-\frac{N+1}{2}\right)^2\\
		&= \frac{1}{N} \frac{N^3-N}{12}\\
		\var(X) &= \frac{N^2-1}{12}
	\end{align*}
	\item Find the mean and variance of the outcome from a single die from the above formulas.
	\begin{align*}
		N &= 6\\
		\mu &= \frac{N+1}{2}\\
		\mu &= \frac{7}{2}\\
		\sigma^2 &= \frac{N^2-1}{12} \\
		\sigma^2 &= \frac{35}{12}
	\end{align*}
\end{enumerate}
\newpage
\section*{Exercise 2.42}
A random variable $X$ has the following probability density function:
\[ f(x) = 
	\left\{ \begin{array}{ll}
		cx(1-x), & 0 \leq x \le 1\\
		0, \text{elsewhere}
	\end{array} \right\} \]
	
\begin{enumerate}[\quad(a)]
	\item Find the constant $c$ so that $f(x)$ is a probability density function.
	
	\begin{align*}
		\int_{-\infty}^{\infty} f(x) &=1 \\
		&= \int_{-\infty}^{0} f(x) + \int_{0}^{1} f(x) + \int_{1}^{\infty} f(x)\\
		&= \int_0^1 cx(1-x) \, dx \\
		&= c\left(\int_0^1 x \, dx - \int_0^1 x^2 \, dx\right)\\
		&= c\left(\tfrac{1}{2} [x^2]_0^1 - \tfrac{1}{3} [x^3]_0^1\right)\\
		&= c\left(\tfrac{1}{2} (1-0) - \tfrac{1}{3} (1-0) \right)\\
		&= c\left(\tfrac{1}{2} - \tfrac{1}{3} \right)\\
		1&= \tfrac{c}{6}\\
		c&= 6
	\end{align*}
	\item Find the cumulative distribution function of $X$.
	\begin{align*}
		0 \leq x \leq 1\\
		\int_0^x f(x) \, dx &= \int_0^x 6\left(x-x^2\right) \, dx\\
		&= 6\left( \int_0^x x\, dx - \int_0^x x^2 \, dx \right)\\
		&= 6\left( \tfrac{1}{2} [x^2]_0^x - \tfrac{1}{3} [x^3]_0^x\right)\\
		&= 6\left( \tfrac{1}{2}[x^2-0] - \tfrac{1}{3}[x^3-0]\right)\\
		&= 6\left( \frac{x^2}{2} - \frac{x^3}{3}\right)\\
		&= 3x^2 - 2x^3\\
		F(x) &= \left\{
		\begin{array}{ll}
			\int_{-\infty}^{x} f(x) \, dx & -\infty \le x \leq 0 \\
			\int_{-\infty}^{0} f(x) \, dx + \int_{0}^{x} f(x) \, dx  & 0 \le x \le 1\\
			\int_{-\infty}^{1} f(x) \, dx + \int_{1}^{x} f(x) \, dx & 3 \le x \le \infty
		\end{array} \right\} \\
		F(x) &= \left\{
		\begin{array}{ll}
		0 & -\infty \le x \leq 0 \\
		\int_{0}^{x} f(x) \, dx  & 0 \le x \le 1\\
		1 & 3 \le x \le \infty
		\end{array} \right\} \\
		F(x) &= \left\{
		\begin{array}{ll}
		0 & -\infty \le x \leq 0 \\
		3x^2 - 2x^3 & 0 \le x \le 1\\
		1 & 3 \le x \le \infty
		\end{array} \right\} \\
	\end{align*}
	\item Find the mean and variance of $X$.
	\begin{align*}
		E(X) &= \int_{-\infty}^{\infty} xf(x)\, dx\\
		&= \int_{-\infty}^{0} xf(x) + \int_{0}^{1} xf(x) + \int_{1}^{\infty} xf(x)\\
		&= \int_0^1 c(x^2-x^3) \, dx \\
		&= 6 \left( \frac{1}{3} [x^3]_0^1 - \frac{1}{4} [x^4]_0^1 \right)\\
		&= 6\left(\frac{1}{3} - \frac{1}{4}\right)\\
		&= \tfrac{6}{12} \\
		E(X) &= \frac{1}{2} \\
		\var(X) &= E(X^2) - \mu^2\\
		&= \int_{-\infty}^{\infty} x^2 f(x)\, dx - \tfrac{1}{4}\\
		&= \int_{-\infty}^{0} x^2f(x) + \int_{0}^{1} x^2f(x) + \int_{1}^{\infty} x^2f(x) - \tfrac{1}{4}\\
		&= \int_0^1 c(x^3-x^4) \, dx - \tfrac{1}{4}\\
		&= 6 \left( \frac{1}{4} [x^4]_0^1 - \frac{1}{5} [x^5]_0^1 \right) - \tfrac{1}{4}\\
		&= 6\left(\frac{1}{4} - \frac{1}{5}\right) - \tfrac{1}{4}\\
		&= \tfrac{6}{20} - \tfrac{1}{4} \\
		\var(X) &= \frac{1}{20} \\
	\end{align*}
\end{enumerate}
\end{document}