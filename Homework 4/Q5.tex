\documentclass{article}
\usepackage{fancyhdr}
\usepackage{amsmath,amssymb}
\usepackage{geometry}
\usepackage{datetime}
\usepackage{enumerate}
\usepackage{graphicx}

%Insert page formatting here
%\hoffset = -.5in
\voffset = -0.375in
%\textwidth = 6in
\textheight = 8in
\headheight = 24pt

\pagestyle{fancy}

\rhead{Peter Olson\\Student ID: $441666$}
\lhead{Math 3200\\Homework 4}
\chead{\today}
\cfoot{}

%\addtolength{\headwidth}{\marginparsep}
%\addtolength{\headwidth}{\marginparwidth}

%\renewcommand{\labelitemi}{$\diamond$}
\renewcommand{\implies}{\rightarrow}
\newcommand{\widespace}{\qquad \qquad \;}
\newcommand{\tret}{\\ \hline}
\newcommand{\fh}{\tfrac{1}{2}}
\newcommand{\deriv}[2]{\frac{d #1}{d #2}}
\newcommand{\pderiv}[2]{\frac{\delta #1}{\delta #2}}
\newcommand{\vr}{\vec{r}}
\newcommand{\at}{\text{ at }}
\newcommand{\var}{\text{Var}}

\begin{document}
\section*{Exercise 2.69}
The \textbf{negative binomial distribution} is a generalization of the geometric distribution. It is the distribution of the random variable $X$ = the number of independent Bernoulli trials, each with success probability $p$, required to obtain $r \geq 1$ successes. It is used as an alternative to the Poisson distribution to model count data.

\begin{enumerate}[\quad(a)]
	\item Show that the distribution of $X$ is given by:
	\[ \binom{x-1}{r-1} p^r (1-p)^{x-r}, x = r, r+1, ... \]
	
	Let us first define a binomial distribution with the variant term ($x$) as the number of trials, and the number of successes constant ($r$):
	\[ f(x) = \binom{x}{r} p^r(1-p)^{x-r} \]
	However, because we are assured of at least one success in the outcomes that we're concerned with, it doesn't make sense to multiply the probability of a certain outcome by $\binom{x}{r}$, rather, $\binom{x-1}{r-1}$ makes sense in this context. Therefore, the expression for the probability distribution function becomes:
	\[ f(x) = \binom{x-1}{r-1} p^r(1-p)^{x-r} \]
	Logically, also, this is only define for the discrete integer $x \geq r$. After all, it's impossible to have $r$ successes with less than $r$ trials.
	
	\item Show that the mean and variance of a negative binomial random variable are:
	\[ E(X) = \frac{r}{p} \quad \text{and} \quad \var(X) = \frac{r(1-p)}{p^2} \]
	\begin{align*}
		E(X) &= E(Y_1) + E(Y_2) + ... + E(Y_n)\\
		E(Y_i) &= 0\left( \frac{1}{1-p} \right) + 1\left( \frac{1}{p} \right)  =\frac{1}{p}\\
		E(X) &= \sum_{i = 0}^r E(Y_i) = \sum_{i = 0}^r \frac{1}{p}\\
		E(X) &= \frac{r}{p}\\
	\end{align*}
	\begin{align*}
		\var(X) &= \sum_{i = 0}^r \var(Y_i)\\
		\var(Y_i) &= E(X^2) - [E(X)]^2\\
		&= \left[ 0^2\left( \frac{1}{1-p} \right) + 1^2\left( \frac{1}{p} \right) \right] - \frac{1}{p^2}\\
		\var(Y_i) &= \frac{1}{p} - \frac{1}{p^2} = \frac{(1-p)}{p^2}\\
		\var(X) &= \sum_{i = 0}^r \frac{(1-p)}{p^2}\\
		\var(X) &= \frac{r(1-p)}{p^2}
	\end{align*}
\end{enumerate}
	
\end{document}