\documentclass{article}
\usepackage{fancyhdr}
\usepackage{amsmath,amssymb}
\usepackage{geometry}
\usepackage{datetime}
\usepackage{enumerate}
\usepackage{graphicx}

%Insert page formatting here
%\hoffset = -.5in
\voffset = -0.375in
%\textwidth = 6in
\textheight = 8in
\headheight = 24pt

\pagestyle{fancy}

\rhead{Peter Olson\\Student ID: $441666$}
\lhead{Math 3200\\Homework 4}
\chead{\today}
\cfoot{}

%\addtolength{\headwidth}{\marginparsep}
%\addtolength{\headwidth}{\marginparwidth}

%\renewcommand{\labelitemi}{$\diamond$}
\renewcommand{\implies}{\rightarrow}
\newcommand{\widespace}{\qquad \qquad \;}
\newcommand{\tret}{\\ \hline}
\newcommand{\fh}{\tfrac{1}{2}}
\newcommand{\deriv}[2]{\frac{d #1}{d #2}}
\newcommand{\pderiv}[2]{\frac{\delta #1}{\delta #2}}
\newcommand{\vr}{\vec{r}}
\newcommand{\at}{\text{ at }}
\newcommand{\var}{\text{Var}}

\begin{document}
\section*{Exercise 2.64}
	A Typist makes a typographical error at the rate of 1 every 20 pages. Let $X$ be the number of errors in a manuscript of 200 pages.
	\begin{enumerate}[\quad(a)]
		\item Write the binomial distribution model for $X$ and show how you will calculate the exact binomial probability that the manuscript has at least 5 errors.
		
		\begin{align*}
			\text{Let } n &= 200\\
			\text{Let } p &= \frac{1}{20} = 0.05\\
			f(x) &= \binom{200}{x} (0.05)^x (0.95)^{200-x}\\
			P(X \leq 5) &= \sum_{x_i = 0}^{5} f(x_i) \\
		\end{align*}
		The rest of the calculation is trivial arithmetic, ergo, I have shown how I would calculate the exact binomial probability that the manuscript has at least 5 errors.
		\item Why is it reasonable to assume that $X$ has a Poisson distribution? Calculate the Poisson approximation to the probability in $(a)$.
		
		It is reasonable to presume a Poisson distribution for $X$ because the Poisson distribution is the binomial distribution taken to the limits of $n \Rightarrow \infty$ and $p \Rightarrow 0$. Because of this, it's used to model the occurrences of rare events, where the opportunity for occurrence is high, but the probability of the event itself occurring is low.
		\begin{align*}
			\lambda &= E(X) = np = 200\times 0.05\\
			\lambda &= 10\\
			f(x) &= \frac{e^{-\lambda}\lambda^x}{x!} = \frac{e^{-10} 10^x}{x!} \\
			P(X \leq 5) &= \sum_{x_i = 0}^{5} f(x_i)\\
			&= \sum_{x_i = 0}^{5} \frac{e^{-10} 10^{x_i}}{x_i!}\\
			P(X \leq 5) &= 0.0670
		\end{align*}
	\end{enumerate}
\end{document}