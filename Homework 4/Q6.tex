\documentclass{article}
\usepackage{fancyhdr}
\usepackage{amsmath,amssymb}
\usepackage{geometry}
\usepackage{datetime}
\usepackage{enumerate}
\usepackage{graphicx}

%Insert page formatting here
%\hoffset = -.5in
\voffset = -0.375in
%\textwidth = 6in
\textheight = 8in
\headheight = 24pt

\pagestyle{fancy}

\rhead{Peter Olson\\Student ID: $441666$}
\lhead{Math 3200\\Homework 4}
\chead{\today}
\cfoot{}

%\addtolength{\headwidth}{\marginparsep}
%\addtolength{\headwidth}{\marginparwidth}

%\renewcommand{\labelitemi}{$\diamond$}
\renewcommand{\implies}{\rightarrow}
\newcommand{\widespace}{\qquad \qquad \;}
\newcommand{\tret}{\\ \hline}
\newcommand{\fh}{\tfrac{1}{2}}
\newcommand{\deriv}[2]{\frac{d #1}{d #2}}
\newcommand{\pderiv}[2]{\frac{\delta #1}{\delta #2}}
\newcommand{\vr}{\vec{r}}
\newcommand{\at}{\text{ at }}
\newcommand{\var}{\text{Var}}

\begin{document}
\section*{Exercise 2.71}
Suppose that the daily high temperature in a particularly temperate place varies uniformly from day to day over the range of $70^\circ$F to $100^\circ$F/ Further assume that the daily temperatures are independently distributed.
\begin{enumerate}[\quad(a)]
	\item What proportion of the days does the daily high temperature exceed $90^\circ$?
	
	Let $X$ be the uniformly distributed temperature over the range $[70,100]$.
	\begin{align*}
		f(x) &= \frac{1}{b-a},\quad a\leq x \leq b\\
		P(X > x) &= 1- F(x)\\
		&= 1 - \frac{x-a}{b-a}\\
		P(X > 90) &= \frac{90-70}{100-70}\\
		P(X > 90) &= \frac{1}{3} = 33.\bar{3}\%
	\end{align*}
	\item Which temperature is exceeded only 10\% of the time?
	\begin{align*}
		1-\frac{x-a}{b-a} &= 0.1\\
		0.9 &= \frac{x-a}{b-a}\\
		0.9(b-a) &= x-a\\
		x &= 0.9(b-a) + a \\
		x &= 0.9\times 30 + 70 = 27+70\\
		x &= 97^\circ\text{F}
	\end{align*}
\end{enumerate}
\end{document}