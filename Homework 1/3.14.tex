\documentclass{article}
\usepackage{fancyhdr}
\usepackage{amsmath,amssymb}
\usepackage{geometry}
\usepackage{datetime}
\usepackage{enumerate}

%Insert page formatting here
\hoffset = -.5in
\voffset = -0.375in
\textwidth = 7in
\textheight = 10in
\headheight = 24pt

\pagestyle{fancy}

\rhead{Peter Olson\\Student ID: $441666$}
\lhead{Math 3200\\Homework 1}
\chead{\today}
\cfoot{}

\renewcommand{\labelitemi}{$\diamond$}
\renewcommand{\implies}{\rightarrow}
\newcommand{\widespace}{\qquad \qquad \;}
\newcommand{\tret}{\\ \hline}
\newcommand{\fh}{\tfrac{1}{2}}
\newcommand{\deriv}[2]{\frac{d #1}{d #2}}
\newcommand{\pderiv}[2]{\frac{\delta #1}{\delta #2}}
\newcommand{\vr}{\vec{r}}
\newcommand{\at}{\text{ at }}

\begin{document}
\begin{minipage}{6in}
\section*{Exercise 3.14}
In each of the following situations, name the sampling method used to divide the data from a list of subjects into a training set and a test set with an allocation of 2/3 and 1/3, respectively,
\begin{enumerate}[\hspace{.5em}a)]
	\item A random number between 1 and 3 is chosen. Starting with that subject, every third subject is assigned to the training set. All others are assigned to the test set.
	
	\hspace{2em} The data is divided into a training and test set with systemic sampling in this case.
	\item A random number is assigned to each subject onthe list using a computer random number generator. Subjects are ordered by their random number, and the first 203 are assigned to the training set. All others are assigned to the test set.
	
	\hspace{2em} Here, the data is divided using random sampling.
	\item Subjects are assigned random numbers in $(b)$ are divided into two lists by gender. With each list, the first 2/3 of the subjects are ordered by random number to the training set. All others are assigned to the test set.
	
	\hspace{2em} This is stratified random sampling to divide the data into two sets.
\end{enumerate}
\end{minipage}

\end{document}