\documentclass{article}
\usepackage{fancyhdr}
\usepackage{amsmath,amssymb}
\usepackage{geometry}
\usepackage{datetime}
\usepackage{enumerate}
\usepackage{graphicx}

%Insert page formatting here
%\hoffset = -.5in
\voffset = -0.375in
%\textwidth = 6in
\textheight = 8in
\headheight = 24pt

\pagestyle{fancy}

\rhead{Peter Olson\\Student ID: $441666$}
\lhead{Math 3200\\Homework 9}
\chead{\today}
\cfoot{}

%\addtolength{\headwidth}{\marginparsep}
%\addtolength{\headwidth}{\marginparwidth}

%\renewcommand{\labelitemi}{$\diamond$}
\renewcommand{\implies}{\rightarrow}
\newcommand{\widespace}{\qquad \qquad \;}
\newcommand{\tret}{\\ \hline}
\newcommand{\fh}{\tfrac{1}{2}}
\newcommand{\deriv}[2]{\frac{d #1}{d #2}}
\newcommand{\pderiv}[2]{\frac{\delta #1}{\delta #2}}
\newcommand{\vr}{\vec{r}}
\newcommand{\at}{\text{ at }}
\newcommand{\var}{\text{Var}}
\newcommand{\cov}{\text{Cov}}

\begin{document}

\section*{Exercise 9.12}

The following data set from a study by the well-known chemist and Nobel Laureate Linus Pauling gives the incidence of cold among 279 French skiers who were randomized tothe Vitamin C and Placebo groups.

\begin{center}
	\begin{tabular}{|r|c|c|c|}
		\hline
		Group & Cold & No cold & Total\\
		\hline
		Vitamin C & 17 & 122 & 139 \\
		\hline
		Placebo & 31 & 109 & 140 \\
		\hline
	\end{tabular}
\end{center}

\begin{enumerate}[\quad(a)]
	\item Test the hypothesis ($ H_0: p_1 = p_2, H_1: p_1 \neq p_2 $) using inference for two proportions.
	\begin{align*}
		\hat{p}_{\text{Vitamin C}} = \hat{p}_v &= x/n_1 & \hat{p}_{\text{Placebo}} = \hat{p}_p &= x/n_1 \\
		\hat{p}_v &= 17/139 & \hat{p}_p &= 31/140\\
	\end{align*}
	\item Test the hypothesis using Pearson chi-square test, and compare your result with part (a).
\end{enumerate}

\end{document}