\documentclass{article}
\usepackage{fancyhdr}
\usepackage{amsmath,amssymb}
\usepackage{geometry}
\usepackage{datetime}
\usepackage{enumerate}
\usepackage{graphicx}

%Insert page formatting here
\hoffset = -.5in
\voffset = -0.375in
\textwidth = 7in
\textheight = 8in
\headheight = 24pt

\pagestyle{fancy}

\rhead{Peter Olson\\Student ID: $441666$}
\lhead{Math 3200\\Homework 9}
\chead{\today}
\cfoot{}

%\addtolength{\headwidth}{\marginparsep}
%\addtolength{\headwidth}{\marginparwidth}

%\renewcommand{\labelitemi}{$\diamond$}
\renewcommand{\implies}{\rightarrow}
\newcommand{\widespace}{\qquad \qquad \;}
\newcommand{\tret}{\\ \hline}
\newcommand{\fh}{\tfrac{1}{2}}
\newcommand{\deriv}[2]{\frac{d #1}{d #2}}
\newcommand{\pderiv}[2]{\frac{\delta #1}{\delta #2}}
\newcommand{\vr}{\vec{r}}
\newcommand{\at}{\text{ at }}
\newcommand{\var}{\text{Var}}
\newcommand{\cov}{\text{Cov}}

\begin{document}

\section*{Exercise 9.20}

\begin{enumerate}[\quad(a)]
	\item State the hypothesis that the four phenotypes appear in the proportion 9:3:3:1.
		$$ H_0 : \quad p_1 = 9/16 \quad p_2 = 3/16 \quad p_3 = 3/16 \quad p_4 = 1/16 $$
	\item Test thte hypotheses. Use $\alpha = 0.05$.
	\begin{align*}
		\hat{e}_n &= n \cdot p_n = 1611 \cdot p_n\\
		\chi^2 &= \sum \frac{(\text{observed } - \text{ expected})^2}{\text{expected}}\\
		&= \frac{(926 - (1611 \cdot (9 / 16)))^2}{(1611 \cdot (9 / 16))} %
		 + \frac{(293 - (1611 \cdot (3 / 16)))^2}{(1611 \cdot (3 / 16))} %
		 + \frac{(288 - (1611 \cdot (3 / 16)))^2}{(1611 \cdot (3 / 16))} %
		 + \frac{(104 - (1611 \cdot (1 / 16)))^2}{(1611 \cdot (1 / 16))}\\
		&= 0.4332 + 0.2719 + 0.6547 + 0.1090 \\
		&= 1.4687\\
		\chi^2_{(4-1),\ 0.05} &= 7.815\\
		\chi^2 &< \chi^2_{3, 0.05}
	\end{align*}
	Because $\chi^2 < \chi^2_{3,\ 0.05}$, we must accept $H_0$, which tells us that the four phenotypes appear in the proportion 9:3:3:1.
\end{enumerate}

\end{document}