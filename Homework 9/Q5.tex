\documentclass{article}
\usepackage{fancyhdr}
\usepackage{amsmath,amssymb}
\usepackage{geometry}
\usepackage{datetime}
\usepackage{enumerate}
\usepackage{graphicx}

%Insert page formatting here
\hoffset = -.5in
\voffset = -0.375in
\textwidth = 7in
\textheight = 8in
\headheight = 24pt

\pagestyle{fancy}

\rhead{Peter Olson\\Student ID: $441666$}
\lhead{Math 3200\\Homework 9}
\chead{\today}
\cfoot{}

%\addtolength{\headwidth}{\marginparsep}
%\addtolength{\headwidth}{\marginparwidth}

%\renewcommand{\labelitemi}{$\diamond$}
\renewcommand{\implies}{\rightarrow}
\newcommand{\widespace}{\qquad \qquad \;}
\newcommand{\tret}{\\ \hline}
\newcommand{\fh}{\tfrac{1}{2}}
\newcommand{\deriv}[2]{\frac{d #1}{d #2}}
\newcommand{\pderiv}[2]{\frac{\delta #1}{\delta #2}}
\newcommand{\vr}{\vec{r}}
\newcommand{\at}{\text{ at }}
\newcommand{\var}{\text{Var}}
\newcommand{\cov}{\text{Cov}}

\begin{document}

\section*{Exercise 9.32}

\begin{enumerate}[\quad(a)]
	\item The sampling is product multinomial.
	\item Test $H_0:\ p_{ij} = P(X = i, Y = j) = P(X = i) P(Y = j) = p_ip_j$

	Given that the R test that I ran returned a p-value of less than $2.2\times 10^{-16}$, we can reject the null hypothesis at almost any level of certainty, and accept that eye color and hair color are associated.

	The test statistic has a distribution of $\chi^2_{9}$.
\section*{R Code}
\begin{verbatim}
# Import data
A <- matrix(
  c(
    68, 20, 15, 5, 
    119, 84, 54, 29, 
    26, 17, 14, 14, 
    7, 94, 10, 16
  ), nrow = 4, ncol = 4)
# Run test
(chisq.test(A))
\end{verbatim}

\section*{R Output}
\begin{verbatim}
> A <- matrix(
+   c(
+     68, 20, 15, 5, 
+     119, 84, 54, 29, 
+     26, 17, 14, 14, 
+     7, 94, 10, 16
+   ), nrow = 4, ncol = 4)

> chisq.test(A)

	Pearson's Chi-squared test

data:  A
X-squared = 138.29, df = 9, p-value < 2.2e-16
\end{verbatim}
\end{enumerate}

\end{document}